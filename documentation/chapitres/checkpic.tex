%!TEX encoding = UTF-8 Unicode
%!TEX root = ../piccolo.tex

\cleardoublepage

\chapterLabel{Construction \texttt{checkpic}}{checkpic}
\index{Mot réservé!checkpic}

%--- Pour supprimer tout en-tête et pied de page sur la 1re page d'un chapitre
\thispagestyle{empty}

\textcolor{red}{\bf Cette construction n'est définie que pour les \emph{pic18}.}

Sa syntaxe est la suivante :

\begin{piccolo}
checkpic "nom", ..., "nom"
\end{piccolo}

C'est-à-dire le mot réservé \pic!checkpic! suivi de une ou plusieurs chaînes de caractères, chacune d'elle désignant un type de micro-contrôleur. Une erreur est déclenchée si le code en cours de compilation n'est pas destiné à l'un des micro-contrôleurs cités.

Par exemple :

\begin{piccolo}
pic18 "18F2480" ...
  ...
checkpic "18F2480", "18F4480"
  ...
\end{piccolo}

Cette écriture sera compilée sans erreur car le type du micro-contrôleur cité dans l'en-tête est le \texttt{18F2480}, et ce nom est l'un de ceux cités dans la construction \pic!checkpic!.

Autre exemple :

\begin{piccolo}
pic18 "18F26K80" ...
  ...
checkpic "18F2480", "18F4480"
  ...
\end{piccolo}

Cette écriture déclenche une erreur de compilation car le type du micro-contrôleur cité dans l'en-tête est le \texttt{18F26K80}, et ce nom n'est pas cité dans la construction \pic!checkpic!.




\section{Utilisation combinée avec \texttt{include}}
L'intérêt de la construction \pic!checkpic! est son utilisation combinée avec l'inclusion de fichier source (construction \pic!include!, \refSectionPage{sectionInclude}).

En effet, l'inclusion de fichier source peut être utilisée pour décrire des routines génériques, valables pour plusieurs types de micro-contrôleurs.

Le code source a alors l'allure suivante :
\begin{piccolo}
pic18 "18F26K80" ...
  ...
include "monCodeGenerique_xxK80.piccolo"
  ...
\end{piccolo}

Mais si le code générique n'est valide que pour les micro-contrôleurs de la même famille \texttt{18F26K80}, \texttt{18F46K80}, \dots\ il n'y a aucune garantie que ce code générique soit utilisé qu'avec ces micro-contrôleurs. Ainsi, si on écrit :

Le code source a alors l'allure suivante :
\begin{piccolo}
pic18 "18F4480" ...
  ...
include "monCodeGenerique_xxK80.piccolo"
  ...
\end{piccolo}

Il n'y a pas obligatoirement de message d'erreur, et un code invalide pour un \texttt{18F4480} est utilisé.


La solution pour déclencher une erreur dans cette dernière situation est d'utiliser la construction \pic!chekpic! dans le fichier inclus \pic!monCodeGenerique_xxK80.piccolo! :

\begin{piccolo}
  ...
checkpic "18F26K80", "18F46K80"
  ...
\end{piccolo}


 