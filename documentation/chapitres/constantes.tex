%!TEX encoding = UTF-8 Unicode
%!TEX root = ../piccolo.tex

\chapterLabel{Définition de constante}{constante}

%--- Pour supprimer tout en-tête et pied de page sur la 1re page d'un chapitre
\thispagestyle{empty}

Des constantes peuvent être définies pour les trois types de programme (\emph{baseline}, \emph{mid-range}, \emph{pic18}).

Une constante est définie par la construction~:

\begin{piccolo}
const nom_constante = expression_immediate
\end{piccolo}


La forme générale d’une \emph{expression immédiate} est décrite à la \refSectionPage{expressionImmediate}. Voici des exemples de définition de constantes~:

\begin{piccolo}
const multiplicateur = 249
const v1 = 128 - (2500 / multiplicateur)
const v2 = 128 + (2500 / multiplicateur)
\end{piccolo}

L’évaluation des expressions s’effectue en 32 bits signé, et doit pouvoir être calculée à la compilation ; elle peut faire référence à d’autres constantes, du moment qu’elles sont définies auparavant dans le programme. Par exemple, l’écriture suivante déclenche une erreur à la compilation~:

\begin{piccolo}
const valeur = mult * 11 # Erreur : mult non defini
const mult = 23 
\end{piccolo}

Vous pouvez utiliser une constante dans toute expression immédiate. Par exemple~:

\begin{piccolo}
if (! PORTB.7)
  if (! PORTB.6)
     movlw butSensTrigo
   else # Sens horaire
     movlw butSensHoraire
  end
end
\end{piccolo}


\sectionLabel{Constantes prédéfinies}{constantesPredefinies}

Piccolo prédéfinit les constantes listées dans le \refTableau{tableauConstantesPredefinies}, en fonction du type du micro-contrôleur. Le fichier listing (obtenu par l'option « \texttt{-L} », voir \refSubsectionPage{listeOptions}) contient la liste des constantes prédéfinies et leur valeur.

\begin{table}[htbp]
  \centering
  \small
  \fondTableau
  \begin{tabular}{llccc}
    \textbf{Nom} & \textbf{Signification} & \textbf{Baseline} & \textbf{Mid-range} & \textbf{Pic18}\\
    \pic!ROM_SIZE! & Taille de la ROM & & \checkmark & \checkmark \\
    \hdashline
    \pic!BOOTLOADER_SIZE! & Taille du \emph{bootloader} & & & \checkmark \\
    \hdashline
    \pic!RAM_SIZE! & Taille de la RAM & & & \checkmark \\
  \end{tabular}
  \caption{Constantes prédéfinies}
  \labelTableau{tableauConstantesPredefinies}
  \ligne
\end{table}

\textbf{Attention !} La valeur de \pic!ROM_SIZE! est le nombre d'octets pour un \emph{pic18} (une instruction occupe 2 ou 4 octets), et le nombre d'instructions de 14 bits pour un \emph{mid-range}.

La constante \pic!BOOTLOADER_SIZE! n’est prédéfinie que pour l’implémentation d’un bootloader ou d’un programme applicatif d'un \emph{pic18}. Elle n’est pas définie pour un programme monolithique. Elle a pour valeur la taille en octets réservée pour l’implémentation du bootloader.

La constante \pic!RAM_SIZE! permet d'écrire un code générique pour \emph{pic18} réalisant l'initialisation à zéro de la RAM.



\subsectionLabel{Exemple d'utilisation de \texttt{ROM\_SIZE}}{exempleLTBLPTR-ROMSIZE}

La constante \pic!ROM_SIZE! est définie pour un \emph{pic18}. Associée avec l'instruction \pic!ltblptr!, elle permet d'écrire une routine générique, valable pour tous les types de \emph{pic18}, qui renvoie dans \pic!WREG!, la somme des octets de la flash.


\begin{piccolo}
routine CALCULER_SOMME_CONTROLE_FLASH {
#--- Entrer l'adresse du dernier mot de la flash
  ltblptr ROM_SIZE - 1
#--- Accumulateur de la somme de controle
  movlw 0
#--- Boucle
  do
   tblrd *-
   addwf TABLAT, W
  while (TBLPTRL NZ | TBLPTRH NZ | TBLPTRU NZ)
  end
}
\end{piccolo}



\subsection{Exemple d'utilisation de \texttt{RAM\_SIZE}}

La constante \pic!RAM_SIZE! est définie pour un \emph{pic18}. Elle permet l'écriture d'une séquence générique mettant à zéro toute la RAM d'un micro-contrôleur. On peut faire figurer cette séquence au début de la routine \emph{main}.


\begin{piccolo}
  lfsr 0, RAM_SIZE - 1
  do
    clrf POSTDEC0
  while (FSR0L NZ | FSR0H NZ)
  end
  clrf INDF0
\end{piccolo}


\sectionLabel{Forme générale d'une expression littérale}{expressionImmediate}

\textbf{Opérandes.} Piccolo définit les opérandes suivants~:
\begin{itemize}
  \item une constante littérale entière, écrite en décimal ou en hexadécimal ;
  \item une constante littérale caractère (qui a pour valeur le code ASCII correspondant) ;
  \item une constante définie par la construction const (voir à cette page) ;
  \item un registre accompagné éventuellement d’un indice ;
  \item une valeur descriptive de registre (\refSectionPage{exprLiteraleDescriptiveDeRegistre}).
\end{itemize}

\textbf{Parenthèses.} Vous pouvez utiliser des parenthèses pour exprimer le groupement.

\textbf{Opérateurs.} Les opérateurs définis pour une expression littérale sont donnés par le \refTableau{operateursExpressionImmediate} (par ordre de priorité croissante).

\begin{table}[htbp]
  \centering
  \small
  \fondTableau
  \begin{tabular}{lll}
    \textbf{Opérateur} & \textbf{Priorité} & \textbf{Description}\\
    \pic!|! & 0 & Ou inclusif bit à bit \\
    \hdashline
    \pic!^! & 0 & Ou exclusif bit à bit \\
    \hdashline
    \pic!&! & 1 & Et bit à bit \\
    \hdashline
    \pic!==! & 2 & Test d'égalité \\
    \hdashline
    \pic+!=+ & 2 & Test d'inégalité \\
    \hdashline
    \pic!>=! & 2 & Test supérieur ou égal \\
    \hdashline
    \pic!>! & 2 & Test strictement supérieur \\
    \hdashline
    \pic!<=! & 2 & Test inférieur ou égal \\
    \hdashline
    \pic!<! & 2 & Test strictement inférieur \\
    \hdashline
    \pic!+! & 3 & Addition \\
    \hdashline
    \pic!<<! & 3 & Décalage à gauche \\
    \hdashline
    \pic!>>! & 3 & Décalage arithmétique à droite \\
    \hdashline
    \pic!*! & 4 & Multiplication \\
    \hdashline
    \pic!/! & 4 & Division \\
    \hdashline
    \pic!%! & 4 & Modulo \\
    \hdashline
    \pic!-! & 5 & Opérateur unaire~: négation \\
    \hdashline
    \pic!~! & 5 & Opérateur unaire~: complémentation logique \\
  \end{tabular}
  \caption{Opérateurs d'une expression littérale, par ordre de priorité croissante}
  \labelTableau{operateursExpressionImmediate}
  \ligne
\end{table}



\sectionLabel{Expression littérale~: valeur descriptive de registre}{exprLiteraleDescriptiveDeRegistre}

Pour illustrer cette construction, nous prenons un exemple. Nous voulons initialiser le registre \assembleur{T2CON} du timer 2 (qui existe sur plusieurs PIC18, dont le 18F448). La documentation du composant nous fournit la description ci-contre. Nous voulons affecter les valeurs suivantes~:\begin{itemize}
  \item \assembleur{TOUTPS} à 5 (1:6 postscale) ;
  \item \assembleur{TMR2ON} à 1 ;
  \item \assembleur{T2CKPS} à 1 (Prescaler is 1).
\end{itemize}

Une première solution est de calculer à la main la valeur à affecter à \assembleur{T2CON}. Cela donne~:
\begin{piccolo}
movlw   0x2D
movwf   T2CON
\end{piccolo}

La seconde solution est d'utiliser la construction que nous allons décrire. Pour cela, il faut d'abord s’assurer que Piccolo connaît la description des champs de bits de \assembleur{T2CON}. On lance donc la commande \texttt{piccolo -R=18F448} dans le terminal, et on recherche la définition du registre \assembleur{T2CON} dans le listing obtenu ; on obtient~:
\begin{quote}
\texttt{\textquotesingle T2CON\textquotesingle~at 0xFCA <-, TOUTPS[4], TMR2ON, T2CKPS[2]>}
\end{quote}

Les noms correspondent bien. En utilisant ces noms, nous pouvons écrire~:
\begin{piccolo}
movlw   T2CON (TOUTPS:5, TMR2ON:1, T2CKPS:1)
movwf   T2CON
\end{piccolo}

L'expression litérale présente la syntaxe suivante~: elle est introduite par un nom de registre\footnote{Avant la version 3.1.0, cette construction était introduite par un un \texttt{@}. À partir de la version 3.1.0, le caractère \texttt{@} permet de définir des \emph{étiquettes} (\refSectionPage{etiquette}).}, puis, entre parenthèses, une séquence de couples formés d’un nom de champ et d’une valeur immédiate~:
\begin{piccolo}
  nom_registre (nom_champ:expression_literale, ...)
\end{piccolo}

Tout registre spécial, toute variable déclarée (dans une section \pic!ram!) ayant des champs définis peut être utilisée.

Le nom de champ (\pic!nom_champ!) est l’un des champs définis par ce registre ou par cette variable. On ne peut utiliser des noms de champ d’un autre registre ou d’une autre variable.

L’\pic!expression_literale! associée obéit à la syntaxe générale des expressions litérales. La valeur obtenue doit être positive ou nulle, et inférieure ou égal à la valeur maximum que peut contenir le champ (c’est à dire, pour un champ de $n$ bits, $2^n-1$).

Un nom de champ ne peut apparaître qu’une fois. L’ordre d’apparition des champs dans cette construction n’a pas d’importance. Tout champ non mentionné a implicitement tous les bits correspondants à zéro.


