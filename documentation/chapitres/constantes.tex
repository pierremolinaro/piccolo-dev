%!TEX encoding = UTF-8 Unicode
%!TEX root = ../piccolo.tex

\cleardoublepage

\chapterLabel{Définition de constante}{constante}

%--- Pour supprimer tout en-tête et pied de page sur la 1re page d'un chapitre
\thispagestyle{empty}

Des constantes peuvent être définies pour les trois types de programme (\emph{baseline}, \emph{mid-range}, \emph{pic18}).

Une constante est définie par la construction :

\begin{lstlisting}[language=piccolo]
const nom_constante = expression_immediate
\end{lstlisting}


La forme générale d’une \emph{expression immédiate} est décrite à la \refSectionPage{expressionImmediate}. Voici des exemples de définition de constantes :

\begin{lstlisting}[language=piccolo]
const multiplicateur = 249
const v1 = 128 - (2500 / multiplicateur)
const v2 = 128 + (2500 / multiplicateur)
\end{lstlisting}

L’évaluation des expressions s’effectue en 32 bits signé, et doit pouvoir être calculée à la compilation ; elle peut faire référence à d’autres constantes, du moment qu’elles sont définies auparavant dans le programme. Par exemple, l’écriture suivante déclenche une erreur à la compilation :

\begin{lstlisting}[language=piccolo]
const valeur = mult * 11 # Erreur : mult non defini
const mult = 23 
\end{lstlisting}

Vous pouvez utiliser une constante dans toute expression immédiate. Par exemple :

\begin{lstlisting}[language=piccolo]
if (! PORTB.7)
  if (! PORTB.6)
     movlw butSensTrigo
   else # Sens horaire
     movlw butSensHoraire
  end
end
\end{lstlisting}


\sectionLabel{Constantes prédéfinies}{constantesPredefinies}

Piccolo prédéfinit les constantes listées dans le \refTableau{tableauConstantesPredefinies}, en fonction du type du micro-contrôleur. Le fichier listing (obtenu par l'option « \texttt{-L} », voir \refSubsectionPage{listeOptions}) contient la liste des constantes prédéfinies et leur valeur.

\begin{table}[ht]
  \centering
  \rowcolors{2}{\fondTableau}{}
  \begin{tabular}{llccc}
    \textbf{Nom} & \textbf{Signification} & \textbf{Baseline} & \textbf{Mid-range} & \textbf{Pic18}\\
    \hline
    \texttt{ROM\_SIZE} & Taille de la ROM & & \checkmark & \checkmark \\
    \texttt{BOOTLOADER\_SIZE} & Taille du \emph{bootloader} & & & \checkmark \\
    \texttt{RAM\_SIZE} & Taille de la RAM & & & \checkmark \\
  \hline
  \end{tabular}
  \caption{Constantes prédéfinies}
  \labelTableau{tableauConstantesPredefinies}
\end{table}

\textbf{Attention !} La valeur de \texttt{ROM\_SIZE} est le nombre d'octets pour un \emph{pic18} (une instruction occupe 2 ou 4 octets), et le nombre d'instructions de 14 bits pour un \emph{mid-range}.

La constante \texttt{BOOTLOADER\_SIZE} n’est prédéfinie que pour l’implémentation d’un bootloader ou d’un programme applicatif d'un \emph{pic18}. Elle n’est pas définie pour un programme monolithique. Elle a pour valeur la taille en octets réservée pour l’implémentation du bootloader.

La constante \texttt{RAM\_SIZE} permet d'écrire un code générique pour \emph{pic18} réalisant l'initialisation à zéro de la RAM.



\subsection{Exemple d'utilisation de \texttt{ROM\_SIZE}}

La constante \texttt{ROM\_SIZE} est définie pour un \emph{pic18}. Associée avec l'instruction \texttt{ltblptr}, elle permet d'écrire une routine générique, valable pour tous les types de \emph{pic18}, qui renvoie dans \emph{W}, la somme des octets de la flash.


\begin{lstlisting}[language=piccolo]
routine CALCULER_SOMME_CONTROLE_FLASH {
#--- Entrer l'adresse du dernier mot de la flash
  ltblptr ROM_SIZE - 1
#--- Accumulateur de la somme de controle
  movlw 0
#--- Boucle
  do
   tblrd *-
   addwf TABLAT, W
  while (TBLPTRU NZ | TBLPTRH NZ | TBLPTRL NZ)
  end
}
\end{lstlisting}



\subsection{Exemple d'utilisation de \texttt{RAM\_SIZE}}

La constante \texttt{RAM\_SIZE} est définie pour un \emph{pic18}. Elle permet l'écriture d'une séquence générique mettant à zéro toute la RAM d'un micro-contrôleur. On peut faire figurer cette séquence au début de la routine \emph{main}.


\begin{lstlisting}[language=piccolo]
  lfsr 0, RAM_SIZE - 1
  do
    clrf POSTDEC0
  while (FSR0L NZ | FSR0H NZ)
  end
  clrf INDF0
\end{lstlisting}


\sectionLabel{Forme générale d'une expression immédiate}{expressionImmediate}