%!TEX encoding = UTF-8 Unicode
%!TEX root = ../piccolo.tex

\cleardoublepage

\chapter{Programmes pour mid-range}

%--- Pour supprimer tout en-tête et pied de page sur la 1re page d'un chapitre
\thispagestyle{empty}




\section{Structure d’un programme pour mid-range}

Un programme Piccolo pour \emph{mid-range} a la structure suivante :

\begin{lstlisting}[language=piccolo]
midrange nom "nom_composant" :
  liste_de_sections
end
\end{lstlisting}


Dans l’en-tête :
\begin{itemize}
  \item le nom « \emph{nom} » est le nom du fichier (sans son extension) qui contient ce texte source ;
  \item le nom du composant « \emph{nom\_composant} » doit être exactement le nom de l’un des composants supportés (pour obtenir la liste des \emph{mid-range} pris en charge, utiliser l’option « \texttt{-{}-midrange} », voir \refSubsectionPage{listeMidrange}).
\end{itemize}


Le corps du programme est constitué d’une liste non ordonnée de sections. Les sections disponibles sont listées dans le \refTableau{sectionsMidrange}.
\begin{table}[ht]
  \centering
  \rowcolors{2}{\fondTableau}{}
  \begin{tabular}{p{5cm}lll}
    \textbf{Type de section} & \textbf{Mot-clé introductif} & \textbf{Référence}\\
    \hline
    Configuration & \texttt{configuration} & \refChapterPage{configuration}\\
    Définition de variable & \texttt{ram} & \refChapterPage{ram}\\
    Définition de constante & \texttt{const} & \refChapterPage{constante}\\
    Définition de routine mid-range & \texttt{routine} & \refSectionPage{routineMidrange}\\
    Définition de routine d'interruption mid-range & \texttt{interrupt} & \refSectionPage{routineInterruptionMidrange}\\
  \hline
  \end{tabular}
  \caption{Les sections d'un programme pour \emph{mid-range}}
  \labelTableau{sectionsMidrange}
\end{table}




\sectionLabel{Routines mid-range}{routineMidrange}




\sectionLabel{Routines d'interruption mid-range}{routineInterruptionMidrange}

