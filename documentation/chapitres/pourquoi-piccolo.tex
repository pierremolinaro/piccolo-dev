%!TEX encoding = UTF-8 Unicode
%!TEX root = ../piccolo.tex


\chapter{Pourquoi j’ai écrit Piccolo}

%--- Pour supprimer tout en-tête et pied de page sur la 1re page d'un chapitre
\thispagestyle{empty}

J’avais écrit un programme d’environ 2400 lignes assembleur pour un PIC 18F448, qui fonctionnait parfaitement. Ce programme n’utilisait que les registres accessibles depuis l’access bank, et j’avais écrit toutes les instructions pour l’accès se fasse de cette façon.

Le programme n’était pas terminé, et j’allais ajouter et code pour l’interface réseau CAN, qui utilise des registres situés dans la page 15. J’ai donc commencé par ajouter au début du programme l’instruction \assembleur{MOVLB 0xF}, qui affecte la valeur 15 au registre de page \assembleur{BSR} (il est initialisé à zéro au démarrage). Là, surprise, plus rien ne fonctionnait. La seule explication, qui s’est avéré exacte par la suite, est que certains accès ne s’effectuaient pas par l’access bank comme je pensais, mais via le registre \assembleur{BSR}. Après plusieurs heures d’examen du code source, j’ai enfin trouvé ce qui n’allait pas, plusieurs instructions du style :

\assembleur{NEGF variable, F}

\assembleur{F} étant un symbole que j’avais initialisé à 1 (en fait, c’était une très mauvaise idée de définir \assembleur{F} !) Or l’instruction \assembleur{NEGF} considère le second argument comme l’indicateur précisant l’accès via le registre \assembleur{BSR} : quand il vaut 1, l’accès se fait par \assembleur{BSR}. J’avais confondu avec les instructions comme \assembleur{DECF}, où le second argument indique si la donnée décrémentée est rangée dans le registre (\assembleur{DECF registre, 1}) ou bien dans \assembleur{W} (\assembleur{DECF registre, 0}).

J’ai alors réalisé que le programme que je projetais de faire, qui devait jongler avec les différents bancs, allait être fragile. Dès lors, je me décidai à concevoir un langage dont le compilateur s’apercevrait des erreurs d’adressage vis à vis du banc sélectionné via \assembleur{BSR}.

C’est là le but premier de Piccolo : {\bf garantir l’exactitude de la valeur de \assembleur{BSR} vis à vis des instructions qui accèdent aux registres}.


Prenons un exemple. On considère trois variables :
\begin{itemize}
  \item \assembleur{var0} désignera une variable accessible via l’access bank, c’est à dire indépendamment de la valeur de \assembleur{BSR} ;
  \item \assembleur{var1} désignera une variable de la page 1, c’est à dire accessible quand \assembleur{BSR} vaut 1 ;
  \item \assembleur{var2} désignera une variable de la page 2, c’est à dire accessible quand \assembleur{BSR} vaut 2.
\end{itemize}

En assembleur, si j’écris :
\begin{quote}
\assembleur{clrf  var0}\\
\assembleur{clrf  var1}\\
\assembleur{clrf  var2}\\
\end{quote}

Il n’y a pas d’erreur d’assemblage, et cependant seule la première instruction est correcte. La deuxième instruction fera un \assembleur{clrf} sur le registre d’adresse \assembleur{varBank1 \% 0x100} (« \assembleur{\%} » signifiant « modulo »), et la troisième sur celui d’adresse \assembleur{varBank2 \% 0x100}.

L’écriture correcte en assembleur est :
\begin{quote}
\assembleur{clrf  var0}\\
\assembleur{movlb 1}\\
\assembleur{clrf  var1, BANKED}\\
\assembleur{movlb 2}\\
\assembleur{clrf  var2, BANKED}\\
\end{quote}

La moindre erreur est fatale. Or, si il est facile de la détecter sur un exemple de trois lignes, cela devient beaucoup plus laborieux quand le programme atteint plusieurs milliers de lignes. La sécurité voudrait que l’on insère une instruction \assembleur{MOVLB} avant chaque accès via \assembleur{BSR}. Mais c’est très inefficace… On préfère souvent initialiser \assembleur{BSR} pour un groupe d’instructions.

En assembleur, que ce passe t-il alors si je décide de passer \assembleur{var1} dans le banc 0 ? Il faut revoir toutes les instructions nommant \assembleur{var1} pour changer la valeur de \assembleur{BSR}. 

Maintenant, voyons comment cette même situation est appréhendée en Piccolo.

Piccolo reprend la plupart des instructions assembleur, aussi la séquence des trois instructions précédentes peut être écrite en Piccolo.

En Piccolo, le code est exécutable est structuré en \emph{routines}, aussi on écrira :
\begin{piccolo}
routine maRoutine {
  clrf  var0
  clrf  var1  # Erreur
  clrf  var2  # Erreur
}
\end{piccolo}

Quelques remarques de détail : \pic!routine! et \pic!clrf! sont des mots réservés de Piccolo, et l’instruction assembleur \assembleur{RETURN} est automatiquement ajoutée à la fin des instructions de la routine par la génération de code.

Que se passe-t'il si on compile un programme contenant cette routine ? En Piccolo, la sémantique de la référence à un registre est la suivante :\begin{itemize}
\item d'abord, l'analyseur sémantique regarde si le registre peut être accédé par l'\emph{access bank} ; si oui, l'instruction est correcte ;
\item si non, l'analyseur sémantique regarde si le registre peut être accédé grâce au registre \assembleur{BSR}.
\end{itemize}

Or ici, aucune indication n'est fournie sur la valeur de \assembleur{BSR} : deux erreurs apparaissent pour les 2$^e$ et 3$^e$ instructions.

On corrige ces erreurs au moyen de l’instruction « \assembleur{banksel} » (comme \assembleur{MOVLB}, elle affecte une valeur à \assembleur{BSR}, mais comme sa sémantique est différente, j’ai préféré utiliser un autre nom). La routine corrigée est :
\begin{piccolo}
routine maRoutine {
  clrf  var0
  banksel 1
  clrf  var1
  banksel 2
  clrf  var2
}
\end{piccolo}


Mais on peut aller plus loin : l’en-tête de la routine ne dit rien au sujet de \assembleur{BSR}, ce qui signifie que la valeur de \assembleur{BSR} peut être quelconque au moment de l’appel, et que la routine peut modifier \assembleur{BSR} à sa guise.

Si maintenant, je change l’en-tête en écrivant :
\begin{piccolo}
routine maRoutine bank:requires 1 {
  clrf  var0
  clrf  var1
  banksel 2
  clrf  var2
}
\end{piccolo}

Cet en-tête signifie que le compilateur Piccolo vérifie que la routine ne soit appelée que lorsque \assembleur{BSR} vaut 1. En contre partie, l’instruction « \assembleur{banksel 1} » est inutile et est enlevée (d’ailleurs sa présence provoquerait l’émission un warning signalant son inutilité).

On retrouve là la programmation par contrat :
\begin{itemize}
  \item obligation de l’appelant : appeler \assembleur{maRoutine} quand \assembleur{BSR} contient 1 ;
  \item bénéfice de l’appelé : pouvoir utiliser directement l’accès au banc 1. 
\end{itemize}

Il y a d’autres possibilités d’écriture de l’en-tête vis à vis de \assembleur{BSR} : elles sont décrites dans le chapitre consacré à ce registre.

Le compilateur Piccolo simule l’évolution de l’exécution du programme vis à vis de \assembleur{BSR}. Ceci a apporté des contraintes dans le design de Piccolo, notamment l’abandon de l’utilisation des instructions de saut et de saut conditionnel pour diriger le flot d’exécution à l’intérieur d’une routine. Piccolo propose à leur place quatre instructions structurées :
\begin{itemize}
  \item l’instruction conditionnelle simple (exploitation directe des instructions sautant conditionnellement l’instruction qui les suit) ;
  \item l’instruction conditionnelle structurée (\pic!if ... elsif ... else ... end!) ;
  \item l’instruction répétitive (\pic!do ... while ... while ... end!) ;
  \item l’instruction répétitive infinie (\pic!forever ... end!).
\end{itemize}

Une conséquence de l’emploi des ces instructions structurées est une augmentation de volume du code engendré : c’est pourquoi un optimiseur est incorporé et permet d’atteindre des résultats très proches d’un code assembleur optimisé écrit à la main.
