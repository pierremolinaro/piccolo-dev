%!TEX encoding = UTF-8 Unicode
%!TEX root = ../piccolo.tex

\cleardoublepage

\chapterLabel{Calcul des besoins en pile et des durées d'exécutions}{pile-et-durees}

%--- Pour supprimer tout en-tête et pied de page sur la 1re page d'un chapitre
\thispagestyle{empty}

Par défaut, la compilation d'un source pour \emph{pic18} lance le calcul des besoins en pile et des durées d'exécutions de chaque routine. Ces résultats apparaissent à la fin du fichier listing. Les options contrôlant cette générations sont décrites à la \refSubsubsectionPage{optionsGeneration}, notamment :
\begin{description}

  \item[\texttt{-L}, \texttt{-{-}list}] Produit un fichier listing contenant de nombreux détails concernant la configuration, l’allocation de la RAM, les optimisations réalisées et la transformation des branchement relatifs en absolu. Ce fichier est écrit dans le même répertoire que le fichier source, porte le même nom, mais avec l’extension \texttt{.list}.

  \item[\texttt{-I}, \texttt{-{-}no-routine-duration}] Par défaut, Piccolo tente de calculer la durée d'exécution des routines. Cette option inhibe ce calcul.

  \item[\texttt{-T}, \texttt{-{-}no-stack-computation}] Par défaut, Piccolo tente de calculer les besoins en pile du programme. Cette option inhibe ce calcul.

  \item[\texttt{-d}, \texttt{-{-}detailled-computations}] Insère dans le fichier listing le détail des calculs de pile et de durée d'exécutions du programme.
\end{description}

\section{Calcul des besoins en pile}

Pour chaque routine, Piccolo essaie de calculer le nombre de niveaux pile dont elle a besoin. Une routine n'appelant aucune routine a un besoin nul.

Évidemment, ce calcul statique est impossible pour une routine récursive, et pour toute routine appelant une routine récursive. Piccolo détecte ces situations et affiche « \emph{recursive} » à la place du nombre de niveaux. 

Si un programme n'a pas de routine d'interruption, le nombre total de niveaux de pile nécessaire est celui de la routine \texttt{main}.


Si un programme a des routines d'interruption, en supposant que celles-ci ne sont pas interruptibles et donc ne s'imbriquent pas, le nombre total de niveaux de pile nécessaire est :
\begin{equation*}
  N_{main} + 1 + \text{sup} \left(N_{\_high\_interrupt\_routine} + N_{\_low\_interrupt\_routine}\right)
\end{equation*}



\section{Calcul des durées d'exécution}

Pour toutes les routines non récursives, Piccolo tente de calculer les bornes de leur durée d'exécution. Il y a une condition supplémentaire, l'absence d'instruction entraînant un bouclage, à savoir l'\emph{instruction de répétition infinie} (\refSectionPage{repetitionInfinie}), et l'\emph{instruction répétitive} (\refSectionPage{instructionRepetitive}). En fonction des instructions conditionnelles, Piccolo calcule le \emph{meilleur} et le \emph{pire} temps d'exécution.

Ces durées sont exprimées en nombre de cycles de l'horloge des instructions : la plupart des instructions s'exécute en un cycle, et quelques instructions comme \texttt{CALL}, \texttt{RETURN}, \texttt{MOVFF}, \dots\ en deux cycles.

Évidemment, aucune durée ne peut être établie pour la routine \texttt{main}, celle-ci se terminant obligatoirement par une boucle infinie. Cependant, ce calcul peut être utile pour les routines d'interruption, à condition qu'elle ne comportent pas de boucle, ce qui est souvent le cas.

\section{Exemple}

Considérons le programme suivant :

\begin{lstlisting}[language=piccolo]
pic18 optimizations "18F2580" :

routine optimize1 {
  if (STATUS.Z)
    movlw 1
  elsif (! STATUS.OV)
    movlw 2
  else
    movlw 3
  end
}

routine optimize2 {
  if (STATUS.Z)
    if (! STATUS.OV)
      movlw 1
    else
      movlw 2
    end
  else
    addlw 3
  end
}

noreturn routine main bank:requires 0 {
  rcall optimize1
  rcall optimize2
  forever
  end
}

end
\end{lstlisting}

Ce programme est juste pris pour exemple de calcul de niveau de pile et de durée d'exécution, le faire tourner n'aurait pas grand intérêt.

\subsection{Résultats en absence d'optimisation}

Sans l'option « \texttt{-O} », le code engendré est le suivant, extrait du fichier listing engendré grâce à l'option « \texttt{-L} ».

La routine \texttt{optimize1} :

\begin{lstlisting}[language=assembleur, frame=l]
    2  000002 :           optimize1:
    3  000002 : A4D8        BTFSS  STATUS, 2
    4  000004 : D002        BRA   _label_0
    5  000006 : 0E01        MOVLW 0x1
    6  000008 : D005        BRA   _label_1
    7  00000A :           _label_0:
    8  00000A : B6D8        BTFSC  STATUS, 3
    9  00000C : D002        BRA   _label_2
   10  00000E : 0E02        MOVLW 0x2
   11  000010 : D001        BRA   _label_3
   12  000012 :           _label_2:
   13  000012 : 0E03        MOVLW 0x3
   14  000014 :           _label_3:
   15  000014 :           _label_1:
   16  000014 : 0012        RETURN
\end{lstlisting}

La routine \texttt{optimize2} :

\begin{lstlisting}[language=assembleur, frame=l]
   17  000016 :           optimize2:
   18  000016 : A4D8        BTFSS  STATUS, 2
   19  000018 : D006        BRA   _label_4
   20  00001A : B6D8        BTFSC  STATUS, 3
   21  00001C : D002        BRA   _label_6
   22  00001E : 0E01        MOVLW 0x1
   23  000020 : D001        BRA   _label_7
   24  000022 :           _label_6:
   25  000022 : 0E02        MOVLW 0x2
   26  000024 :           _label_7:
   27  000024 : D001        BRA   _label_5
   28  000026 :           _label_4:
   29  000026 : 0F03        ADDLW 0x3
   30  000028 :           _label_5:
   31  000028 : 0012        RETURN
   32  00002A :           main:
\end{lstlisting}

Les besoins en pile sont faciles à calculer, ces routines n'appelant pas d'autres routines et étant appelées par \texttt{main}. Le résultat calculé par Piccolo apparaît dans le fichier listing sous la forme suivante :
\begin{lstlisting}[language=assembleur, frame=l]
Stack needs:
                                        Routine Levels
                                      optimize1 0
                                      optimize2 0
                                           main 1
\end{lstlisting}

L'option « \texttt{-d} » permet d'insérer dans le fichier listing le détail des calculs des durées minimum et maximum de chaque routine, exprimés en nombre de cycles de l'horloge des instructions (la plupart des instructions s'exécute en un cycle, et quelques instructions comme \texttt{CALL}, \texttt{RETURN}, \texttt{MOVFF}, \dots\ en deux cycles). Pour la routine \texttt{optimize1}, on obtient :

\begin{lstlisting}[language=assembleur, frame=l]
 Line:  min-  max:
    2:    0-    0:LABEL optimize1
    3:    1-    1:BTFSS STATUS, 2
    4:    3-    3:BRA _label_0
    5:    3-    3:MOVLW 0x1
    6:    5-    5:BRA _label_1
    7:    3-    3:LABEL _label_0
    8:    4-    4:BTFSC STATUS, 3
    9:    6-    6:BRA _label_2
   10:    6-    6:MOVLW 0x2
   11:    8-    8:BRA _label_3
   12:    6-    6:LABEL _label_2
   13:    7-    7:MOVLW 0x3
   14:    7-    8:LABEL _label_3
   15:    5-    8:LABEL _label_1
   16:    7-   10:RETURN
optimize1 --> 7 ... 10
\end{lstlisting}

Pour chaque ligne, sont affichés les instants \emph{minimum} et \emph{maximum} auxquels se termine l'exécution de l'instruction citée. La dernière ligne récapitule résultat : la routine \texttt{optimize1} s'exécute au mieux en 7 cycles de l'horloge des instructions, au pire en 10.

Et pour la routine \texttt{optimize2}, on obtient :

\begin{lstlisting}[language=assembleur, frame=l]
   17:    0-    0:LABEL optimize2
   18:    1-    1:BTFSS STATUS, 2
   19:    3-    3:BRA _label_4
   20:    3-    3:BTFSC STATUS, 3
   21:    5-    5:BRA _label_6
   22:    5-    5:MOVLW 0x1
   23:    7-    7:BRA _label_7
   24:    5-    5:LABEL _label_6
   25:    6-    6:MOVLW 0x2
   26:    6-    7:LABEL _label_7
   27:    8-    9:BRA _label_5
   28:    3-    3:LABEL _label_4
   29:    4-    4:ADDLW 0x3
   30:    4-    9:LABEL _label_5
   31:    6-   11:RETURN
optimize2 --> 6 ... 11
\end{lstlisting}

On n'obtient aucun résultat pour la routine \texttt{main}, car celle-ci boucle.

\subsection{Résultats avec optimisation}

L'optimisation (option « \texttt{-O} ») transforme la séquence \texttt{MOVLW k - RETURN} en \texttt{RETLW k}, et un branchement vers une instruction \texttt{RETURN} en \texttt{RETURN}. La routine \texttt{optimize1} devient donc :

\begin{lstlisting}[language=assembleur, frame=l]
 Line:  min-  max:
    2:    0-    0:LABEL optimize1
    3:    1-    1:BTFSS STATUS, 2
    4:    3-    3:BRA _label_0
    5:    4-    4:RETLW 0x1
    7:    3-    3:LABEL _label_0
    8:    4-    4:BTFSC STATUS, 3
    9:    6-    6:RETLW 0x3
   10:    7-    7:RETLW 0x2
optimize1 --> 4 ... 7
\end{lstlisting}


Et la routine \texttt{optimize2} :

\begin{lstlisting}[language=assembleur, frame=l]
 Line:  min-  max:
   17:    0-    0:LABEL optimize2
   18:    1-    1:BTFSS STATUS, 2
   19:    3-    3:BRA _label_4
   20:    3-    3:BTFSC STATUS, 3
   21:    5-    5:RETLW 0x2
   22:    6-    6:RETLW 0x1
   28:    3-    3:LABEL _label_4
   29:    4-    4:ADDLW 0x3
   30:    4-    4:-
   31:    6-    6:RETURN
optimize2 --> 5 ... 6
\end{lstlisting}


