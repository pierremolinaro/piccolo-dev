%!TEX encoding = UTF-8 Unicode
%!TEX root = ../piccolo.tex


\chapterLabel{Calcul des besoins en pile et des durées d'exécution}{pile-et-durees}

%--- Pour supprimer tout en-tête et pied de page sur la 1re page d'un chapitre
\thispagestyle{empty}

\textcolor{red}{\bf Uniquement pour les \emph{pic18}, à partir de la version 3.1.1.}

Par défaut, la compilation d'un source pour \emph{pic18} lance le calcul des besoins en pile et des durées d'exécutions de chaque routine. Ces résultats apparaissent à la fin du fichier listing. Les options contrôlant cette génération sont décrites à la \refSubsubsectionPage{optionsGeneration}, notamment :
\begin{description}

  \item[\texttt{-L}, \texttt{-{-}list}] Produit un fichier listing contenant de nombreux détails concernant la configuration, l’allocation de la RAM, les optimisations réalisées et la transformation des branchement relatifs en absolu. Ce fichier est écrit dans le même répertoire que le fichier source, porte le même nom, mais avec l’extension \texttt{.list}.
\end{description}

\section{Calcul des besoins en pile}

Pour chaque routine, Piccolo essaie de calculer le nombre de niveaux pile dont elle a besoin. Une routine n'appelant aucune routine a un besoin nul.

Évidemment, ce calcul statique est impossible pour une routine récursive, et pour toute routine appelant une routine récursive. Piccolo détecte ces situations et affiche « \emph{recursive} » à la place du nombre de niveaux. 

Si un programme n'a pas de routine d'interruption, le nombre total de niveaux de pile nécessaire est celui de la routine \pic!main!.


Si un programme a des routines d'interruption, en supposant que celles-ci ne sont pas interruptibles et donc ne s'imbriquent pas, le nombre total de niveaux de pile nécessaire est :
\begin{equation}
  N_{main} + 1 + \text{sup} \left(N_{high\_interrupt\_routine} + N_{low\_interrupt\_routine}\right)
\end{equation}

où :
\begin{itemize}
\item $N_{main}$ est le nombre de niveaux de pile requis par le programme principal ;
\item $N_{high\_interrupt\_routine}$ est le nombre de niveaux de pile requis par le sous-programme d'interruption de haute priorité ;
\item $N_{low\_interrupt\_routine}$ est le nombre de niveaux de pile requis par le sous-programme d'interruption de basse priorité.
\end{itemize}


\section{Calcul des durées d'exécution}

Pour toutes les routines non récursives, Piccolo tente de calculer les bornes de leur durée d'exécution. Il y a une condition supplémentaire, l'absence d'instruction entraînant un bouclage, à savoir l'\emph{instruction de répétition infinie} (\refSectionPage{repetitionInfinie}), et l'\emph{instruction répétitive} (\refSectionPage{instructionRepetitive}), ou un agencement de l'instruction \pic!block! (\refSectionPage{instructionBloc}) entraînant un bouclage. En fonction des instructions conditionnelles, Piccolo calcule le \emph{meilleur} et le \emph{pire} temps d'exécution.

Ces durées sont exprimées en nombre de cycles de l'horloge des instructions : la plupart des instructions s'exécute en un cycle, et quelques instructions comme \assembleur{CALL}, \assembleur{RETURN}, \assembleur{MOVFF}, \dots\ en deux cycles.

Évidemment, aucune durée ne peut être établie pour la routine \pic!main!, celle-ci se terminant obligatoirement par une boucle infinie. Cependant, ce calcul peut être utile pour les routines d'interruption, à condition qu'elle ne comportent pas de boucle, ce qui est souvent le cas.

\section{Exemple}

À titre d'exemple, considérons le programme d'exemple \emph{blink\_led\_it\_18F448.piccolo}, où les commentaires, la déclaration des variables et la configuration ont été retirés :

\begin{piccolo}
pic18 blink_led_it_18F448 "18F448" :

...

interrupt high fast {
  BCF  PIR1.TMR2IF
  banksel 2
  if (compteurL NZ)
    DECF compteurL
  elsif (compteurH NZ)
    DECF compteurH
    SETF compteurL
  else
  #--- Reinitialiser les compteurs
    MOVLW  0x13
    MOVWF  compteurH
    MOVLW  0x87
    MOVWF  compteurL
  #--- Clignoter
    BTG  PORTE.0
  end
}

noreturn routine main bank:requires 0 {
  MOVLW 7
  MOVWF ADCON1
  BCF  TRISE.0
  banksel 2
  CLRF compteurL
  CLRF compteurH
  MOVLW  250 - 1 # La periode est PR2 + 1
  MOVWF  PR2
  MOVLW  T2CON (TOUTPS:0, TMR2ON:1, T2CKPS:1) # New syntax in 3.1.0
  MOVWF  T2CON  
  BSF  RCON.IPEN
  BSF  PIE1.TMR2IE
  BSF  INTCON.GIEH
  forever
  end
}

end
\end{piccolo}


\subsection{Taille de la pile}

Le programme principal et le sous-programme d'interruption n'appellent pas de sous-programmes : le besoin en pile est donc de $1$ niveau, celui du sous-programme d'interruption. Ce résultat apparaît sur la sortie console, si l'option \emph{verbose output} (« \texttt{-v} ») est activée :

\begin{lstlisting}[language=assembleur]
...
Max stack depth: 1
...
\end{lstlisting}

Un résultat détaillé est disponible dans le fichier listing \emph{blink\_led\_it\_18F448.list} qui est produit si l'option « \texttt{-L} » est présente :

\begin{lstlisting}[language=assembleur]
...
*******************************************************************
*                       STACK COMPUTATIONS                        *
*******************************************************************

Levels Routine
     0 .HIGH_INTERRUPT
     0 .START

...
\end{lstlisting}



\subsection{Durées d'exécution}

On suppose que l'optimisation (option « \texttt{-O} ») est activée, ainsi que l'option « \texttt{-L} » qui produit le fichier listing dont les codes suivants sont extraits. Le code intermédiaire engendré pour le sous-programme d'interruption est : 

\begin{lstlisting}[language=assembleur]
LABEL .HIGH_INTERRUPT, ORG 0x8:
  (JUMP _high_interrupt)

LABEL _high_interrupt:
  BCF PIR1, 1
  MOVLB 0x2
  compteurL Z ? (JUMP .L1) : JUMP .L0

LABEL .L1:
  compteurH Z ? (JUMP .L4) : JUMP .L3

LABEL .L4:
  MOVLW 0x13
  MOVWF compteurH
  MOVLW 0x87
  MOVWF compteurL
  BTG PORTE, 0
  RETFIE FAST

LABEL .L0:
  DECF compteurL, F, BSR_ACCESS
  RETFIE FAST

LABEL .L3:
  DECF compteurH, F, BSR_ACCESS
  SETF compteurL
  RETFIE FAST
\end{lstlisting}

Pour calculer les bornes des durées d'exécution du sous-programme d'interruption, Piccolo commence par la fin, c'est-à-dire les blocs dont il peut directement calculer les durées ; ici, ce sont les blocs \texttt{.L4}, \texttt{.L0} et \texttt{.L3} :

\begin{lstlisting}[language=assembleur]
Addr.  Code       Duration    Assembly
00014                         .L4:
00014  0E13       7-7             MOVLW 0x13
00016  6F01       6-6             MOVWF compteurH
00018  0E87       5-5             MOVLW 0x87
0001A  6F00       4-4             MOVWF compteurL
0001C  7084       3-3             BTG PORTE, 0
0001E  0011       2-2             RETFIE FAST

Addr.  Code       Duration    Assembly
00020                         .L0:
00020  0700       3-3             DECF compteurL, F, BSR_ACCESS
00022  0011       2-2             RETFIE FAST

Addr.  Code       Duration    Assembly
00024                         .L3:
00024  0701       4-4             DECF compteurH, F, BSR_ACCESS
00026  6900       3-3             SETF compteurL
00028  0011       2-2             RETFIE FAST
\end{lstlisting}

Les durées d'exécution sont affichées sous la forme \emph{min-max}, et l'unité de temps est une période de l'horloge des instructions. Les durées affichées sont les durées d'exécution à partir de ligne courante jusqu'à la fin de l'exécution du sous-programme d'interruption ; par exemple, pour le label \texttt{.L0}, la durée est de $3$ : 1 cycle d'horloge pour l'instruction \texttt{DECF} et 2 cycles pour \texttt{RETFIE}. Les bornes \emph{min} et \emph{max} sont égales car le code est linéaire, sans branchement.

Comme les durées de \texttt{.L4} et \texttt{.L3} sont calculées, Piccolo calcule la durée de \texttt{.L1} :
\begin{lstlisting}[language=assembleur]
LABEL .L1:
  compteurH Z ? (JUMP .L4) : JUMP .L3
\end{lstlisting}

Rappelons que \texttt{JUMP .L4} est placé entre parenthèses pour signifier que l'instruction de saut n'est pas engendrée, l'optimiseur ayant placé le bloc \texttt{.L4} à la suite du bloc \texttt{.L1}.

Le résultat est :
\begin{lstlisting}[language=assembleur]
Addr.  Code       Duration    Assembly
00010                         .L1:
00010  6701       7-9             TSTFSZ compteurH, BSR_ACCESS
00012  D008                       BRA .L3
\end{lstlisting}

Voyons comment les bornes 7 et 9 sont calculées :
\begin{itemize}
\item si \texttt{compteurH} est nul, le test est vrai, et l'instruction \texttt{BRA .L3} est ignorée, ce qui prend 2 cycles ; comme le bloc \texttt{.L4} s'exécute en 7 cycles, la durée est donc de 9 cycles ;
\item si \texttt{compteurH} est non nul, le test est faux (1 cycle), et l'instruction \texttt{BRA .L3} est exécutée (2 cycles) ; comme le bloc \texttt{.L3} s'exécute en 4 cycles, la durée est donc de 7 cycles.
\end{itemize}

La connaissance des bornes de \texttt{.L1} (et de \texttt{.L0}) permet de calculer les bornes de \texttt{\_high\_interrupt} :
\begin{lstlisting}[language=assembleur]
LABEL _high_interrupt:
  BCF PIR1, 1
  MOVLB 0x2
  compteurL Z ? (JUMP .L1) : JUMP .L0
\end{lstlisting}

Le résultat est :
\begin{lstlisting}[language=assembleur]
Addr.  Code       Duration    Assembly
00008                         _high_interrupt:
00008  929E       8-13            BCF PIR1, 1
0000A  0102       7-12            MOVLB 0x2
0000C  6700       6-11            TSTFSZ compteurL, BSR_ACCESS
0000E  D008                       BRA .L0
\end{lstlisting}

Les bornes 6 et 11 du terminateur sont calculées de la façon suivante :
\begin{itemize}
\item si \texttt{compteurL} est nul, le test est vrai, et l'instruction \texttt{BRA .L0} est ignorée (2 cycles) ; comme le bloc \texttt{.L1} s'exécute en 7 à 9 cycles, la durée est donc de 9 à 11 cycles ;
\item si \texttt{compteurL} est non nul, le test est faux (1 cycle), et l'instruction \texttt{BRA .L0} est exécutée (2 cycles) ; comme le bloc \texttt{.L0} s'exécute en 3 cycles, la durée est donc de 6 cycles.
\end{itemize}

La borne min de la durée d'exécution du terminateur est donc de 6 cycles, et la borne max 11 cycles.

Pour obtenir les bornes de la durée d'exécution du sous-programme d'interruption, il suffit d'ajouter les durées des deux instructions \texttt{BCF} et \texttt{MOVLB} (1 cycle chacune). On obtient ainsi les bornes 8 et 13.






















%\subsection{Résultats en absence d'optimisation}
%
%Sans l'option « \texttt{-O} », le code engendré est le suivant, extrait du fichier listing engendré grâce à l'option « \texttt{-L} ».
%
%La routine \assembleur{optimize1} :
%
%\begin{lstlisting}[language=assembleur, frame=l]
%    2  000002 :           optimize1:
%    3  000002 : A4D8        BTFSS  STATUS, 2
%    4  000004 : D002        BRA   _label_0
%    5  000006 : 0E01        MOVLW 0x1
%    6  000008 : D005        BRA   _label_1
%    7  00000A :           _label_0:
%    8  00000A : B6D8        BTFSC  STATUS, 3
%    9  00000C : D002        BRA   _label_2
%   10  00000E : 0E02        MOVLW 0x2
%   11  000010 : D001        BRA   _label_3
%   12  000012 :           _label_2:
%   13  000012 : 0E03        MOVLW 0x3
%   14  000014 :           _label_3:
%   15  000014 :           _label_1:
%   16  000014 : 0012        RETURN
%\end{lstlisting}
%
%La routine \assembleur{optimize2} :
%
%\begin{lstlisting}[language=assembleur, frame=l]
%   17  000016 :           optimize2:
%   18  000016 : A4D8        BTFSS  STATUS, 2
%   19  000018 : D006        BRA   _label_4
%   20  00001A : B6D8        BTFSC  STATUS, 3
%   21  00001C : D002        BRA   _label_6
%   22  00001E : 0E01        MOVLW 0x1
%   23  000020 : D001        BRA   _label_7
%   24  000022 :           _label_6:
%   25  000022 : 0E02        MOVLW 0x2
%   26  000024 :           _label_7:
%   27  000024 : D001        BRA   _label_5
%   28  000026 :           _label_4:
%   29  000026 : 0F03        ADDLW 0x3
%   30  000028 :           _label_5:
%   31  000028 : 0012        RETURN
%   32  00002A :           main:
%\end{lstlisting}
%
%Les besoins en pile sont faciles à calculer, ces routines n'appelant pas d'autres routines et étant appelées par \pic!main!. Le résultat calculé par Piccolo apparaît dans le fichier listing sous la forme suivante :
%\begin{lstlisting}[language=assembleur, frame=l]
%Stack needs:
%                                        Routine Levels
%                                      optimize1 0
%                                      optimize2 0
%                                           main 1
%\end{lstlisting}
%
%L'option « \texttt{-d} » permet d'insérer dans le fichier listing le détail des calculs des durées minimum et maximum de chaque routine, exprimés en nombre de cycles de l'horloge des instructions (la plupart des instructions s'exécute en un cycle, et quelques instructions comme \assembleur{CALL}, \assembleur{RETURN}, \assembleur{MOVFF}, \dots\ en deux cycles). Pour la routine \assembleur{optimize1}, on obtient :
%
%\begin{lstlisting}[language=assembleur, frame=l]
% Line:  min-  max:
%    2:    0-    0:LABEL optimize1
%    3:    1-    1:BTFSS STATUS, 2
%    4:    3-    3:BRA _label_0
%    5:    3-    3:MOVLW 0x1
%    6:    5-    5:BRA _label_1
%    7:    3-    3:LABEL _label_0
%    8:    4-    4:BTFSC STATUS, 3
%    9:    6-    6:BRA _label_2
%   10:    6-    6:MOVLW 0x2
%   11:    8-    8:BRA _label_3
%   12:    6-    6:LABEL _label_2
%   13:    7-    7:MOVLW 0x3
%   14:    7-    8:LABEL _label_3
%   15:    5-    8:LABEL _label_1
%   16:    7-   10:RETURN
%optimize1 --> 7 ... 10
%\end{lstlisting}
%
%Pour chaque ligne, sont affichés les instants \emph{minimum} et \emph{maximum} auxquels se termine l'exécution de l'instruction citée. La dernière ligne récapitule résultat : la routine \assembleur{optimize1} s'exécute au mieux en 7 cycles de l'horloge des instructions, au pire en 10.
%
%Et pour la routine \assembleur{optimize2}, on obtient :
%
%\begin{lstlisting}[language=assembleur, frame=l]
%   17:    0-    0:LABEL optimize2
%   18:    1-    1:BTFSS STATUS, 2
%   19:    3-    3:BRA _label_4
%   20:    3-    3:BTFSC STATUS, 3
%   21:    5-    5:BRA _label_6
%   22:    5-    5:MOVLW 0x1
%   23:    7-    7:BRA _label_7
%   24:    5-    5:LABEL _label_6
%   25:    6-    6:MOVLW 0x2
%   26:    6-    7:LABEL _label_7
%   27:    8-    9:BRA _label_5
%   28:    3-    3:LABEL _label_4
%   29:    4-    4:ADDLW 0x3
%   30:    4-    9:LABEL _label_5
%   31:    6-   11:RETURN
%optimize2 --> 6 ... 11
%\end{lstlisting}
%
%On n'obtient aucun résultat pour la routine \assembleur{main}, car celle-ci boucle.
%
%\subsection{Résultats avec optimisation}
%
%L'optimisation (option « \texttt{-O} ») transforme la séquence \assembleur{MOVLW k - RETURN} en \assembleur{RETLW k}, et un branchement vers une instruction \assembleur{RETURN} en \assembleur{RETURN}. La routine \assembleur{optimize1} devient donc :
%
%\begin{lstlisting}[language=assembleur, frame=l]
% Line:  min-  max:
%    2:    0-    0:LABEL optimize1
%    3:    1-    1:BTFSS STATUS, 2
%    4:    3-    3:BRA _label_0
%    5:    4-    4:RETLW 0x1
%    7:    3-    3:LABEL _label_0
%    8:    4-    4:BTFSC STATUS, 3
%    9:    6-    6:RETLW 0x3
%   10:    7-    7:RETLW 0x2
%optimize1 --> 4 ... 7
%\end{lstlisting}
%
%
%Et la routine \assembleur{optimize2} :
%
%\begin{lstlisting}[language=assembleur, frame=l]
% Line:  min-  max:
%   17:    0-    0:LABEL optimize2
%   18:    1-    1:BTFSS STATUS, 2
%   19:    3-    3:BRA _label_4
%   20:    3-    3:BTFSC STATUS, 3
%   21:    5-    5:RETLW 0x2
%   22:    6-    6:RETLW 0x1
%   28:    3-    3:LABEL _label_4
%   29:    4-    4:ADDLW 0x3
%   30:    4-    4:-
%   31:    6-    6:RETURN
%optimize2 --> 5 ... 6
%\end{lstlisting}


